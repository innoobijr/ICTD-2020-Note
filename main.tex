\documentclass{articlei}
\us1::efefe:i

ii
iixqwedfe2epackage[utf8]{inputenc}

\title{ICTD 2020 Note}
\author{Innocent Ndubuisi-Obi}
\date{December 2019}
\begin{document}

\maketitle

\section{Introduction}
Themes
\begin{enumerate}
	\item Interaction between the built and digital environment
		\item Problems and solutions
		\item dynamics and how they manifest in 'very new' problems
		\item the need ot thing of social and urban information architecture in solutions
	\item Ubiquity of mobile computing devices
	\item High costs of data and connectivity
	\item Dense and Recurrent mobility behaviors
	\item Inequality and varying levels of access to digital goods and services
\end{itemize}

David Clark profiles in his recent book "Designs for an Internet" the various factors that influence the design of the inital internet protocols and architectures. He offers a list a value requirements that we must take into account as we design this new internet. The value requirements when combined with resources/constraint lay the foundation to an internet model. 

What we are simply asking in this research project is: what does a sustainable 'internet' model look like for communities that lack access to internet services. In this way, we are not just speaking of HTTP, but a broader host of services and applicaiton supported by a variety of protocols and infrastructure that we call the internet. 

What roles? What services? What garauntees? What methods of verifications? What methods of authentication? And most importantly, to what scale?

Albert Hisrchman on said "One must be immersed in the particulars to discovers anything general". In this light, we will be deconstrcuting the work of Dye and attempting to pull out value specifications: service, actions, and relationships. Then from this activity, present models of how various internets can be configured.

We are starting from a point of view that does not privilege today internet infrastrucutre, but instead start from a place of services, poeple, and relations and from their presenting refications of those systems in the form of an "internet"
\section{Related Work}
\subsection{Device to Device Transfers}
\subsection{Resource-constrained networking}
\subsection{Offline community-based internets}
\subsection{Designing for resource constraint environments}
\section{Research Challenges}
\begin{itemize}
  \item Can we generate a set of design choices that can guide the development and design of offline internets?
  \item Can we discover kernels of hybrid ICT and social computing systems that are scale invariant?
  \item If so, how can we build a sustainable framework of hybrid networking that leverage such a kernel aand local dynamics to support community internets?:
\end{itemize}
\section{Evaluation Plan}
\section{References}
\end{document}
